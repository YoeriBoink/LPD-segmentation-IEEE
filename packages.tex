% !TEX root = paper.tex

%%%%%%%%%%%%
% Packages %
%%%%%%%%%%%%
\usepackage{subfiles}
\usepackage{algpseudocode}
\usepackage{arydshln}

%% Page Layout
\usepackage{fullpage} %http://en.wikibooks.org/wiki/LaTeX/Page_Layout
% Manual headings and margins:
%\usepackage[activate=normal]{pdfcprot} % Textränder werden einheitlicher angepasst, glatter Rand --> deprecated !!!
%\usepackage{fancyhdr}        % definiere einfache Headings
%\usepackage{float}
%\usepackage{vmargin}         % Seitenränder einstellen leichtgemacht
%\usepackage[width=1cm, left=3cm]{geometry} % Alternative für Seitenränder
%\usepackage[Bjornstrup]{fncychap}       % Fancy Chapter Headings


%% PDF properties
\usepackage[pagebackref,pdftex,bookmarks,colorlinks,breaklinks]{hyperref}
% Define properties for the hyperref package
\hypersetup{%
		%colorlinks=true,       %default: false
		linkcolor=magenta,      %has effects if colorlinks=true
		citecolor=blue,				
		filecolor=dullmagenta,
		urlcolor=magenta,
		%menucolor=blue,
		%plainpages=false,
		%pdfpagemode=none,      %Alternative: =FullScreen
		%pdfview=FitV,
		%pdfstartview=FitH,
		%pdffitwindow=true,
		%pdfmenubar=false,      %default true
		%pdfwindowui=false,     %default true
		%bookmarksopen=true,
		%bookmarksnumbered=true,
		%bookmarksopenlevel=2,
}


%% Font packages
\usepackage[T1]{fontenc}    % Zeichensatzkodierung T1
\usepackage[utf8]{inputenc} % Definition des Eingabeencodings, also wie Umlaute umgesetzt werden
\usepackage{lmodern}
\usepackage{ae}             % Verwendung von AE-Fonts
%\usepackage[german,ngerman]{babel}
%\usepackage{color}
%\definecolor{myred}{rgb}{0.695,0.148,0}
%\definecolor{mygray}{gray}{0.6}


%% AMS Latex packages
\usepackage{amsmath,amstext,amsfonts,amssymb,amsthm,latexsym}
% Additional symbols:
%\usepackage{wasysym}
%\usepackage{dsfont}          % Mathematische Zeichen wie R oder C mit Doppelstrichen 


%% TODO notes
\usepackage{todonotes}


%% Figures
\usepackage{graphicx}         % for using includegraphics
\usepackage{epsfig}           % for using epsfig <-- uses includegraphics
\usepackage{subcaption}
%\usepackage{caption,subcaption} % http://en.wikibooks.org/wiki/LaTeX/Floats,_Figures_and_Captions


%% Tables
%\usepackage{longtable}       % seitenübergreifende Tabellen
%\usepackage{array}           % fuer aufwändigere Tabellen
%\usepackage{tabularx}        % tables with fixed col width and row height
%\usepackage{colortbl}        % farbige Tabellen (v. D. Carlisle)
%\usepackage{enumerate,enumitem,multirow,multicol}


%% Code visualization
%\usepackage{listings}        % fuer die Code-Umgebung
% algorithmicx packages:
\usepackage{algorithm}
\usepackage{algpseudocode}
%\usepackage{algorithmic}
%\usepackage[ruled,vlined]{algorithm2e} % do this BEFORE loading algorithmicx


%% Misc
%\usepackage[T1]{url}         % T1: allow line breaks for URLs
%\usepackage{acronym}         % Acronyms
%\usepackage{times}
%\usepackage{pgf,pgfarrows,pgfnodes,pgfautomata,pgfheaps} %Pfeile ?
%\usepackage{background}
%\usepackage{ifthen}
%\usepackage[rflt]{floatflt}
%\usepackage{cancel}			 % fuer Durchstreichungen
\usepackage{bbm}


%%%%%%%%%%%%%%%%%%%%%%%%%%%%%%%%%%%%%%%%%%%%%%%%%%%%%%%

%% Diagrams
%\usepackage{tikz}
%\usetikzlibrary{calc,trees,positioning,arrows,chains,shapes.geometric,%
%    decorations.pathreplacing,decorations.pathmorphing,shapes,%
%    matrix,shapes.symbols}
%\tikzset{
%	>=stealth',
%  element1/.style={
%    rectangle, 
%    rounded corners, 
%    % fill=black!10,
%    draw=black,% thick,
%    text width=10em,
%    minimum height=3em,
%    text centered, 
%    on chain},   
%%   line/.style={draw, thick, <->},
%%  element2/.style={
%%    tape,
%%    top color=white,
%%    bottom color=blue!50!black!60!,
%%    minimum width=8em,
%%    draw=blue!40!black!90, very thick,
%%    text width=10em, 
%%    minimum height=3.5em, 
%%    text centered, 
%%    on chain},
%  every join/.style={->, shorten >=1pt, shorten <=1pt,},% thick},
%  %decoration={brace},
%  %tuborg/.style={decorate},
%  %tubnode/.style={midway, right=2pt},
%}